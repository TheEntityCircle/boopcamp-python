\documentclass[a4paper, 12pt]{article}

\usepackage[T2A]{fontenc}
\usepackage[english, russian]{babel}

\usepackage{indentfirst}

% Пакет для расширения возможностей
% работы с формулами
\usepackage{amsmath}

% Расширение набора
% математических символов
\usepackage{amsfonts}
\usepackage{amssymb}

\begin{document}
\section{Запись формул}
Посреди текста формулу можно включить таким образом: $a = 15x$, блаблабла бла бла блаблаблабла бла бла. Если же нужно выделить формулу на отдельной строке, то используется следующий синтаксис:
\[ X = 0.1 a. \]

Такого же результата можно добиться, используя окружение <<equation*>>:

\begin{equation*}
X = 0.1 a.
\end{equation*}

Если же мы хотим формулу пронумеровать, то будем использовать без звёздочки:

\begin{equation}
I = UR.
\end{equation}

Используя в коде <<label>> мы сможем легко сослаться на нужную формулу по её номеру, например, на формулу \eqref{our_formula}.

\begin{equation}\label{our_formula}
E = mgh.
\end{equation}

\section{Часто встречаемые функции при работе с формулами}
Степень и индекс:
\[ a^2 = a_x^2 + a^2_y + a_z^2, \]
\[ a^{15xyz}_{zxy} .\]

Дроби:
\[ \frac{12}{13}. \]

Красивые скобки и знак умножения:
\[ \left( \frac{a^b}{c_d} \right) = \left[ 15 \cdot \frac{1}{2} \right]. \]

С обычными скобками было бы так:
\[ ( \frac{a^b}{c_d} ) = [ 15 \cdot \frac{1}{2} ].  \]

Функции (для сравнения -- слева использование без команды):
\[ sin(x) + \sin(x) = \ln(5) - \frac{15}{\arccos(0.04)} \cdot \sqrt[6]{10}. \]

Необычные шрифты (для этого, в частности, нам нужен был <<amsfonts>>):
\[ \mathbb{A} + \mathit{D} + \mathbf{T}. \]

Греческие символы (тильда в формуле явно задаёт пробел между символами):
\[ \lambda,~\Delta,~\delta,~\varepsilon,~\omega. \]

Суммы и интегралы:
\[ \sum^{5000}_{n = 1},~ \int,~ \int^1_0, \int\limits^1_0\]

Математические значки:
\[ \Rightarrow \infty \triangle \times \div \in \]

% Полный список символов можно нуйти тут: https://www.overleaf.com/learn/latex/List_of_Greek_letters_and_math_symbols

Значок вектора, точка сверху и двойная точка:
\[ \vec{a}, \dot{a}, \ddot{a} \]



\end{document}
