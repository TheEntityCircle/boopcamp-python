\documentclass[a4paper, 12pt]{article}

\usepackage[T2A]{fontenc}
\usepackage[english, russian]{babel}

\begin{document}

Первая строка.
Вторая строка в файле. (Она на самом деле длинная, не короткая)

Третья строка в файле.

\underline{Это -- подчёркнутый текст.} \textbf{Это -- жирный текст.} \textit{Это -- курсив.}

Тире -- это двойной минус на клавиатуре.

Кавычки ставятся знаками <<больше>> и <<меньше>>.

Размер шрифта можно менять соответствующими командами, а фигурными скобками можно разграничивать область изменения текста: {\LARGE раз два Три.} Затем: {\tiny три Семь.}

% Подробно все модификаторы можно найти в статье: \url{https://www.overleaf.com/learn/latex/Questions/How_do_I_adjust_the_font_size}

\end{document}
