\documentclass[a4paper, 12pt]{article}

\usepackage[T2A]{fontenc}
\usepackage[english, russian]{babel}

\usepackage{indentfirst}


\begin{document}
\section{Вставка таблиц}

Вставим простую таблицу (в редакторе TeXMaker удобно использовать вкладку Помощник -> ,быстрая таблица, также \\ удобен сайт https://tablesgenerator.com/):

\begin{tabular}{|c|c|}
\hline 
Раз & $\frac{1}{51}$ \\  \hline 
Два & 5700 \\  \hline 
\end{tabular} 

Таблица с разным размером столбцов и разными выравниваниями внутри столбца:
\begin{itemize}
	\item p -- фиксированная ширина столбца, выравнивание слева,
	\item l -- автоматическая ширина столбца, выравнивание слева,
	\item r -- аналогично l, но справа,
	\item с -- автоматическая ширина, выравнивание по центру,
\end{itemize}
сама таблица выравнена по центру:
\begin{center}
\begin{tabular}{|p{2cm}|l|r|c|p{90pt}|p{0.08\textwidth}|}
\hline 
7 & 9 & 15129914 & 52 & 42 & 555 \\ 
\hline 
0 & 12412444 & 33 & 2 & 9103920 & 1000 \\ 
\hline 
\end{tabular} 
\end{center}

\section{Красивые таблицы}
И на созданную таким образом таблицу \ref{our_table} можно сослаться, аналогично ссылкам на рисунки:

\begin{table}[h!]
    \centering
    \caption{Тут можно подписать таблицу.}\label{our_table}
	\begin{tabular}{ |c|c|c|}
 \hline
 Значение 1 & 15 & 11 \\ \hline
 Значение 2 & 13 & 14\\ \hline
 Значение 3 & 15 & 017 \\ \hline
 Значение 8 & 15 & -2 \\ \hline
	\end{tabular}
\end{table}

Можно сделать и таблицу без разделителей строк:
\begin{table}[h!]
    \centering
    \caption{Без разделителей строк.}
	\begin{tabular}{ |c|c|c|}
 \hline
 Значение 1 & 15 & 11 \\
 Значение 2 & 13 & 14\\ 
 Значение 3 & 15 & 017 \\ 
 Значение 8 & 15 & -2 \\ \hline
	\end{tabular}
\end{table}

\newpage
Аналогично со столбцами:
\begin{table}[h!]
    \centering
    \caption{Без разделителей столбцов.}
	\begin{tabular}{ |ccc|}
 \hline
 Значение 1 & 15 & 11 \\ \hline
 Значение 2 & 13 & 14\\ \hline
 Значение 3 & 15 & 017 \\ \hline
 Значение 8 & 15 & -2 \\ \hline
	\end{tabular}
\end{table}


Внешние края тоже можно убрать:
\begin{table}[h!]
    \centering
    \caption{Тут можно подписать таблицу.}
	\begin{tabular}{ ccc}
 Значение 1 & 15 & 11 \\ 
 Значение 2 & 13 & 14\\ 
 Значение 3 & 15 & 017 \\ 
 Значение 8 & 15 & -2 \\
	\end{tabular}
\end{table}

\end{document}
